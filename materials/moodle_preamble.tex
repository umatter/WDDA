% ---------------------------------------------------------
% Preamble for WDDA Moodle module-overview document
% ---------------------------------------------------------

\usepackage{booktabs}
\usepackage{colortbl}
\usepackage{enumitem}
\usepackage{xcolor}
\usepackage{hyperref}

% ---- Session block header ----
% #1 = session week number, #2 = topic, #3 = date
\newcommand{\SessionBlock}[3]{%
  \noindent\rule{\textwidth}{0.4pt}\\[0.3em]%
  {\large\textbf{Vorlesungswoche #1:} #2}\hfill\textit{#3}\\[0.3em]%
}

% ---- Material table (with dataset) ----
% #1 = slide-deck base name  (e.g. WDDA\_Lecture\_01)
% #2 = lecture base name      (e.g. lecture\_01)
% #3 = data file              (e.g. WDDA\_01.xlsx)
\newcommand{\MaterialTable}[3]{%
  \smallskip
  \begin{tabular}{@{}p{3cm}l@{}}
  Folien/Handout & \texttt{#1.pdf} \\
  Notizen        & \texttt{notes\_#2.html} \\
  R-Skript       & \texttt{#2.R} \\
  Datensatz      & \texttt{#3} \\
  \end{tabular}\\[0.5em]%
}

% ---- Material table (without notes) ----
% #1 = slide-deck base name
% #2 = lecture base name
% #3 = data file
\newcommand{\MaterialTableNoNotes}[3]{%
  \smallskip
  \begin{tabular}{@{}p{3cm}l@{}}
  Folien/Handout & \texttt{#1.pdf} \\
  R-Skript       & \texttt{#2.R} \\
  Datensatz      & \texttt{#3} \\
  \end{tabular}\\[0.5em]%
}

% ---- Material table (without dataset) ----
% #1 = slide-deck base name
% #2 = lecture base name
\newcommand{\MaterialTableNoData}[2]{%
  \smallskip
  \begin{tabular}{@{}p{3cm}l@{}}
  Folien/Handout & \texttt{#1.pdf} \\
  Notizen        & \texttt{notes\_#2.html} \\
  R-Skript       & \texttt{#2.R} \\
  \end{tabular}\\[0.5em]%
}
